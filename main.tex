\documentclass[stu,12pt]{apa7}
  \usepackage{times}               % Times New Roman Font Face
  \usepackage[american]{babel}     % Localization
  \usepackage[utf8]{inputenc}      % Input Encoding
  \usepackage{hyperref}            % Hyperlinks
  \usepackage{enumitem}            % Additional Enumeration Environment Settings
  \usepackage{geometry}            % Page Layout
  \usepackage{soul}                % Text Highlighting
  \usepackage{graphicx}            % Images
  \usepackage{csquotes}            % Quoting Environment
  \usepackage{bookmark}            % Required by `csquotes'
  \usepackage{mdframed}            % Colorful Tex-Box Environment
  \usepackage[toc]{appendix}       % Appendix
  \usepackage{fancyhdr}            % Headings and Footers
  \usepackage[%
    style=apa,%
    sortcites=true,%
    sorting=nyt%
  ]{biblatex}
  \usepackage{xcolor}

  % Bibliography Setup
  %% Language Mappings
  \DeclareLanguageMapping{english}{english-apa}
  \DeclareLanguageMapping{american}{american-apa}
  %% Bibliography File Path
  \addbibresource{main.bib}
  %% Categories for Specified Bibliography Items
  %%% Category for sources not referenced in-text
  \DeclareBibliographyCategory{consulted}
  \addtocategory{consulted}{noauthor_business_nodate}
  \addtocategory{consulted}{noauthor_college_nodate}
  \addtocategory{consulted}{cuddy_your_2012}
  \addtocategory{consulted}{anonymous_changing_1999}
  \addtocategory{consulted}{paturel_talk_2012}


  % Hyperlink Setup
  \hypersetup{
    colorlinks = true,
    urlcolor = blue,
    linkcolor = blue,
    citecolor = blue
  }

  % Page and Text Layout
  \geometry{%
    a4paper,%
    top=1in,%
    bottom=1in,%
    left=1in,%
    right=1in%
  }
  \setlength{\headheight}{15pt}


  % Title Page
  \title{%
    What Are You Really Saying?
  }
  \shorttitle{Module 4 Essay Assignment}
  \author{Ashton Hellwig}
  \authorsaffiliations{Department of Mathematics, Front Range Community College}
  \course{COM125: Interpersonal Communication}
  \professor{Richard Thomas}
  \duedate{November 29, 2020 23:59:59 MDT}
  \date{\today}
  \lhead{COM125CG1-M4E}
  \abstract{%
    \textbf{Overview}\\%
    As we discovered in Module 2, conflict is everywhere and can definitely
      impact our relationships. Even when we are in agreement with what we say,
      our nonverbal communication might be giving us away.\\%

    In the Module 4 movie clip titled ``This Is 40'', which appears immediately
      below the topic titled ``\hl{The Theme}'', we observe a couple who appear
      to be agreeing on some goals for the future, but their body language may
      be telling us something that their words are not. In addition to analyzing
      the movie clip, consider how nonverbal communication affects one or more
      of your significant relationships. View the other video clips as well.\\%

    You should spend approximately 6.5 hours on this assignment.%
  }


\begin{document}
  % Title Page
  \maketitle


  \section{How We Utilize Verbal vs. Non-Verbal Communication}
    Verbal and nonverbal communication may be used in different scenarios
      when required, but \(frac{9}{10}\) times, one cannot be occurring without
      the other right along side it
      \parencite[pp. 150]{noauthor_communication_2013}.

    \subsection{The Most Common Facial Expressions and Gestures}
      There are 42 muscles in an adult human's face, and any combination of
        the way they move can yield a different facial expression. To keep the
        list short, we will instead discuss what goes \textit{into} creating
        a facial expression. More prevalent with the current 2020 pandemic,
        eyebrow position is an incredible indicator at a persons emotion as well
        as the eyes themselves. Avoiding eye with tears welling up means
        something entirely different (sadness, disappointment) than tears
        welling up with eyebrows aggravated and a finger waging at someone
        \parencite[pp. 207]{cohn_observer-based_2007}. Even across different
        cultures in other countries, the response to the same situation can
        elicit differing facial expressions! For example, in Japan when annoyed
        many people will simply smile and endure any sort of conversation to
        show their strength and fortitude versus that of the Western countries
        where disgust is almost always painted on one's face when they are
        experiencing it based on another individual's story
        \parencite[pp. 2]{safdar_variations_2009}.

    \subsection{Non-Verbal Communication's Effect on the Conversation}
      In personal relationships, body language is a real expression of what
        someone may be afraid to say out loud. Again, this author would prefer
        to not delve too deep into ``significant relationships'' (assuming
        that is to mean \textit{the relationships between you and your
        significant other}), as these details are a bit of a sore spot at
        present time. Instead, we will focus on other relationships which hold
        just as much appearance in our day to day lives as our significant
        others do: our coworkers and peers.

      In recent times, the inherent fact that we are all required to wear a
        facial covering/face mask in public when outside of our own homes has
        lent itself to an astonishing development in how we communicate
        non-verbally with one-another. Because the mask \textbf{is supposed}
        to cover both our nose and our mouth, the usual cues many people look
        for are now no longer apparent (such as smiling, lip curling, pouting,
        et cetera). Instead, we now focus a lot of our emotion through the use
        of our eyes (or perhaps we always did --- only now it is all we can
        see).

      The face alone tells many things about a person's true emotions they are
        harboring when talking to another person, and are \textit{notoriously}
        difficult to manipulate at one's own will. Even organizations such as
        the United States of America's Federal Bureau of Investigation (FBI)
        utilizes a quantitative system for distinguishing another individual's,
        be it peer, person of interest, or suspect, ``true' intentions by
        assigning numerical values to predefined movements of the facial muscles
        as well as a letter grade to signify its intensity, yielding a
        rudimentary understanding of what someone is conveying through their
        body language which may match or contrast the words spoken by them
        directly \parencite[pp. 209]{cohn_observer-based_2007}.


  \section{Pete and Debbie}
    \subsection{How Does Body Language Changes Throughout the Conversation}
      When the scene begins we can notice how Debbie's body language is very
        explanatory and, as someone would say, inviting. She is being very
        animate by talking with her hands and attempting to explain her side
        of the argument in detail while looking directly at Pete to ensure
        that he is taking in all of the information that is being
        given \parencite{apatow_this_2012}. Pete is obviously more reserved in
        the beginning, which is a sign that he either does not want to be
        \textit{having} this conversation at all, or rather that he just simply
        does not like where he believes the conclusion is heading.

      As the scene of the movie progresses, we can see a small shift in the
        body language being expressed by both individuals in the conversation.
        Pete begins to show a more ``stand-offish'' front (crossing his arms,
        pacing around the room, avoiding eye contact, licking his lips). Debbie,
        on the other hand, went from the animate amazing speaker we initially
        say shift into more of an annoyed demeanor (shoulder shrugging at his
        replies, audible sighing, staying incredibly stationary in her seat)
        \parencite{apatow_this_2012}.

    \subsection{Applying the ``Triangle of Meaning'' To Their Conversation}
      Placeholder.

    \subsection{Determining How Information is Conveyed within the Scene}
      Because TV and movies are a \textit{visual} medium, I believe that the
        writers and directors of this program were intending to make the
        audience feel the full \textbf{weight} of these harder to hear
        conversations by utilizing body language. This also provides a more
        relatable feeling for the viewer, as in many of our own personal
        relationships we have topics we feel strongly about yet are afraid to
        speak about out loud. Our bodies clearly do not like us hiding anything
        as it expresses our true emotions visually for anyone to be able to
        pick up on!


  % Bibliography
  %% Works Cited
  \newpage
  \printbibliography[%
    title={References},%
    heading={bibintoc},%
    notcategory={consulted}%
  ]


  %% Works Consulted
  \newpage
  \nocite{*}
  \printbibliography[%
    title={Additional References},%
    heading={bibintoc},%
    category={consulted}%
  ]
\end{document}
